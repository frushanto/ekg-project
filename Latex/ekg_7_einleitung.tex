%ekg_7_Einleitung und Motivation

\section{Einleitung und Motivation}

Seit der Entdeckung der Methode zur Ableitung der elektrischen Potenziale am menschlichen Herzen, hat die Diagnostik durch das Elektrokardiogramm (EKG) eine essentielle Bedeutung in Kliniken, Arztpraxen und im Rettungsdienst eingenommen. Für den Anwender ist es eine einfache, schnelle und vor allem nicht invasive Methode sich ein Bild vom Zustand der Erregungsleitung am Herzen des Patienten zu machen. Schwerwiegende Erkrankungen wie Infarkte oder Kammerflimmern können sofort diagnostiziert werden, wodurch die Therapie zeitnah eingeleitet werden kann. Doch auch in der Diagnostik von leichteren oder chronisch verlaufenden Krankheiten ist das EKG ein wichtiges Werkzeug im Repertoire des Arztes, wenn der Patient die Gelegenheit kommt bei dem Arzt vorstellig zu werden. Die wenigsten Patienten lassen sich ohne Symptome oder Leidensdruck rein prophylaktisch von ihrem Arzt untersuchen. Dabei beginnen die meisten schweren Erkrankungen mit einem symptomlosen Stadium, in welchem eine einfache Behandlung, zum Beispiel mit Medikamenten und ohne bleibende Schäden möglich wäre. So ist es auch im Fall der Volkskrankheit des Vorhofflimmerns von der in Deutschland etwa 300.000 Menschen betroffen sind. 

Im gesunden Herzen arbeiten Vorhöfe und Kammern zeitlich genau abgestimmt zusammen. Die Hauptlast der Pumpleistung übernehmen die Kammern, diese können ihr volles Potential jedoch nur ausschöpfen, wenn die Vorhöfe kurz vor der Kammerkontraktion kontrahieren. Dass bewirkt die vollständige Füllung der Kammer und eine optimale Ausnutzung der Schlagkraft. Außerdem fließt das Blut so möglichst laminar, also ohne Verwirbelungen. 

Beim Vorhofflimmern kommt es zu einer unvollständigen Kontraktion der Vorhöfe. Durch eine gestörte Erregungsleitung am Herzen arbeiten die Muskelzellen nicht mehr synchron und die Vorhöfe flimmern (manchmal spricht man auch vom "Flattern") nur noch, anstatt koordiniert zu kontrahieren. Die Kammer kann nicht mehr effizient arbeiten und das Blut bildet Turbulenzen. Zum einen führt dies zu einer verminderten Leistungsfähigkeit des Patienten, was in einem Teil der Fälle jedoch nicht bemerkt wird, zum anderen können durch die Verwirbelungen im Blut kleine Thromben entstehen. Diese Thromben, also Blutgerinnsel, werden aus der Kammer in den Blutkreislauf ausgeworfen und können je nach Herzseite im Kapillargebiet der Lunge oder des Gehirns hängen bleiben. So verstopfen sie die Blutzufuhr und das dahinter liegende Gewebe stirbt ab, im Falle vom Gehirn spricht man von einem Schlaganfall (Apoplex), der meist irreparable Schäden nach sich zieht. 

In unserem Projekt wollen wir ein EKG-Gerät entwickeln, dass die Diagnose solcher stiller Herzerkrankungen ermöglicht, ohne dass der Patient sein Zuhause verlassen muss. Das Gerät soll die einfachste Funktionalität erfüllen, das heißt ein Ein-Kanal-EKG (hierfür werden zwei Elektroden am Körper benötigt), und die Daten auf einer Speicherkarte sichern, um sie später durch einen Mediziner auswerten zu lassen. Zusätzlich kann der Anwender während der Messung an einem Display Herzfrequenz und Kurvenverlauf verfolgen. 





