%Realisierung/Akkumanagement und Versorgungsspannungen

\setlength{\parindent}{0em} 

\subsection{Akkumanagement und Versorgungsspannungen}

\subsubsection{Betrachtung der nötigen Energie}

Dieses Unterkapitel behandelt die Analyse der Anforderungen an die Spannungsversorgung eines mobilen EKG Gerätes und erklärt deren Umsetzung.

Im ersten Schritt muss die zu erreichende Laufzeit festgelegt werden. Da das eine Langzeit-EKG Aufnahme möglich sein soll, welche für gewöhnlich 24 Stunden dauert, soll die Versorgung auf 30 Stunden Gerätelaufzeit ausgelegt werden. \\

Um die nötige Kapazität des Akkus festlegen zu können, müssen zunächst sämtliche Hardwarekomponenten mit ihrem durchschnittlichen Verbrauch betrachtet werden. Dies Ergibt sich wie folgt:
\\

\begin{tabular}[h]{l|c|r}
Komponente & Nennspannung & Stromverbrauch während Langzeitaufnahme\\
\hline
Display & 5V & 10mA \\
Bluetooth & 5V & 8mA \\
Cardreader & 5V & 15mA \\
MCU & 3V & 4mA \\
Signalfilterung & 3V & 3mA \\
\end{tabular}
\\
Dadurch ergibt sich eine Leistungsaufnahme von:\\
$ P_{con} = 5V * (10mA + 8mA + 15mA) + 3V * (4mA + 3mA) = 165mW + 21mW = 186mW $

Die Effizienz der noch unbekannten Spannungswandler wird vorläufig mit 80\% angenommen:\\
$P_{draw} = 186mW * \frac{1}{0.8} = 232.5mW $

Multipliziert man den Leistungsbedarf mit der gewünschten Laufzeit so ergibt sich eine nötige Energiemenge von:\\
$W_{Akku} = 232.5mW * 30h = 6975mWh$

\subsubsection{Betrachtung der Zellchemie}

%TODO Quelle für Energiedichte von Zellen suchen

Da Lithium-Ionen Zellen aktuell die höchste Energiedichte für unseren Anwendungsfall liefern, wird diese Art der Zellchemie zu Versorgung des EKG Gerätes verwendet. Um Gefährdungen bei einem Medizinischen Produkt soweit wie möglich vorzubeugen, wird konkret eine Lithium-Polymer-Akku verwendet, welcher Aufgrund der kunststoffähnlichen Eigenschaften des Elektrolyts einen höheren Explosionsschutz sowie bessere Auslaufeigenschaften bietet.
Die Nennspannung dieser Zellen beträgt in der Regel 3,7V. Dadurch ergibt sich eine nötige Ladungsmenge von 
$Q = \frac{6975mWh}{3.7V} = 2051mAh $
Diese Ladungsmenge wird großflächig über die Zellgröße 18560 angeboten, welche Weltweit in Massenfertigung produziert wird und somit keine Finanziellen oder Logistische Probleme darstellt.

%TODO von uns gewählte Zelle beschreiben

\subsubsection{Erzeugung der Unterspannungen}

Eine Li-Po Zelle nimmt während ihrer Entladung Spannungen zwischen 4.2V und 3.2V abhängig vom Ladezustand an. Da sich aus diesem weiten Bereich die Komponenten des EKG-Gerätes nicht zuverlässig versorgen lassen, müssen stabile Zwischenspannungen erzeugt werden. Die erzeugten Spannungen müssen in der Lage sein den maximalen Strom für ihre Baugruppen zu liefern, welcher sich wie folgt zusammensetzt:\\

\begin{tabular}[h]{l|c|r}
Komponente & Nennspannung & Stromverbrauch maximal\\
\hline
Display & 5V & 100mA \\
Bluetooth & 5V & 40mA \\
Cardreader & 5V & 100mA \\
MCU & 3V & 7mA \\
Signalfilterung & 3V & 3mA \\
\end{tabular}
\\
\\
Dies ergibt eine Stromaufnahme von: $I_{3V} = 10mA$ und $I_{5V} = 240mA$
Hierfür bietet sich eine Vielzahl an Möglichkeiten an, welche im Folgendem erläutert werden.\\

Die umfassendste Variante ist die Verwendung eines PMIC (Power Management Integrated Circuit), bei welchem es sich um eine integrierte Schaltung handelt, die alle Anfallenden Aufgaben der Spannungserzeugung übernimmt. Dazu zählen: Battery Management (Überwachung des Ladungszustands der Batterie), Spannungsregulation (Bereitstellen von verschiedenen Unterspannungen), Ladefunktionen. Was auf den ersten Blick als gute Lösung für die gegebenen Anforderungen erscheint, gestaltet sich in der praktischen Umsetzung jedoch schwierig. PMICs kommen idr. im QFN48 Package, welches vergleichsweise groß ist und schwierig zu löten. Somit gestaltet sich das Testen einer Schaltung, welche auf einem PMIC basiert als kompliziert. Hinzu kommen die Vergleichsweise hohen kosten des ICs sowie ein hoher Aufwand an externer Beschaltung. Des weiteren bietet ein PMIC wesentlich mehr Features als für die Projektanforderungen nötig wären, weshalb diese Möglichkeit ausgeschlossen wurde.\\

Als nächste Möglichkeit wurde die Verwendung von Buck- und Boostkonvertern untersucht. Hierbei handelt es sich um integrierte Schaltungen, welche durch zerhacken einer Gleichspannung mittels Transistoren, nutzen der Selbstinduktionseffekte einer Spule sowie anschließende Glättung und Speicherung durch Kondensatoren eine DCDC Wandlung auf höhere oder niedrigere Spannungen ermöglicht. Diese ICs sind verglichen mit PMICs klein, da sie bereits im SOT23 Package erhältlich sind, und können problemlos Ströme im einstelligen Amperebereich bereitstellen. 
Diese Schaltungsart wurde als nächstes mit Bauteilen von Analog Devices in LTSPICE simmuliert. Dabei stellte sich die Erzeugung von 5V durch einen Boost-Konverter als hervorragende Realisierung heraus.
%TODO Spice Bilder einfügen

Bei der Auswahl eines Boost-Konvertes ist vor allem auf ein geringen Eigenverbrauch des ICs sowie eine hohe Schaltfrequenz zu achten. Darüber hinaus gibt es Boost-Konverter mit einem integrierten Enable Pin, welcher nicht nur den IC deaktiviert, sondern auch die Last vollständig vom Eingang abkoppelt. Dies ist überaus nützlich um im Standby Strom zu sparen. Ein Konverter der all diese Anforderungen erfüllt, ist der RP402N501F-TR-FE, welcher bereits ab 0,54€ im Falle einer Massefertigung erhältlich wäre und Ströme bis 800mA unterstützt.

%TODO Spice Bild Restwelligkeit

Die Erzeugung von 3V durch einen Buck-Konverter ist zwar auch ohne weiteres möglich, allerdings weist diese Schaltungsart bauartbedingt immer eine gewisse Restwelligkeit auf. Da die komplette Analogschaltung zur Aufnahme es EKG Signals mit 3V versorgt wird, sollte hier jede Form von Schwankungen oder Ungenauigkeiten dezimiert werden.

