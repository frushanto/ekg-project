%EKG-7 Android-App
\subsection{Android-App}
Dieses Unterkapitel behandelt den Entwurf und die Entwicklung der Home-EKG Android App zur Darstellung eines EKG Signals unter Verwendung der Bluetooth 2.0 Technologie. Die Kommunikation mit einer Datenbank ermöglicht es der App in Echtzeit Patientendaten zu speichern, sowie Authentifizierungsabfragen durchzuführen. Das Projekt wurde hauptsächlich mit der objektorientierten Programmiersprache Java realisiert. 

\subsubsection{Datenbank}
Um eine geräteübergreifende Funktionalität aller Features der Android App zu gewährleisten wurde bei der Entwicklung auf die Echtzeitdatenbank Firebase zurückgegriffen. Die von Google bereitgestellte IDE Android Studio ermöglicht eine besonders einfache Implementierung der verschiedenen Funktionalitäten der Datenbank wie Authentifizierung und Cloud-Speicherung. \\
Bei der Registrierung eines Benutzers wird ein neuer Child-Node im User-Baum angelegt und mit einer individuellen Identifikationsnummer ausgestattet. Über diese kann die App mit dem Server kommunizieren und Daten austauschen. \\
Der Benutzer hat außerdem die Möglichkeit ein Profilbild im .png Format hochzuladen, welches in einen von Firebase verwalteten externen Speicher abgelegt wird. Durch die Verwendung von Google Analytics kann ferner das Nutzerverhalten statistisch erfasst und auf verschiedene Arten ausgewertet werden. 

\subsubsection{Aktivitäten}

\begin{wrapfigure}[11]{R}{0.35\textwidth}
\vspace{-25pt}
\includegraphics[width=0.35\textwidth]{app_activity.png}
\caption{Lifecycle}
\label{app_activity}
\end{wrapfigure}

Jede Activity Klasse stellt eine Bildschirm Seite (Login Screen, Register Screen, Home Screen, etc.) der App dar, mit welcher der User interagieren und somit Befehle an das Betriebssystem weitergeben kann. Beim Start einer neuen Aktivität wird eine Instanz der zugrunde liegenden Java Klasse erzeugt und der zugehörige Lebenszyklus (Abbildung \ref{app_activity}) gestartet.
Um zwischen den drei Hauptzuständen

\begin{itemize}
\item[•] Aktivität läuft im Vordergrund und hat Fokus
\item[•] Aktivität läuft im Hintergrund
\item[•] Aktivität wurde gestoppt
\end{itemize}

zu wechseln, bietet die Activity Klasse außerdem einen Satz von sechs Callback an, die bei Übergängen von verschiedenen Phasen des Lebenszyklus aufgerufen werden. Durch Verwendung der onCreate() Methode können so zum Beispiel Buttons oder Textfelder initialisiert werden.
Die Home-EKG App besteht insgesamt aus sieben Aktivitäten, welche im Folgenden einzeln vorgestellt werden.

\paragraph{Login und Registrierung Aktivität}
\begin{figure} [!h]
	\begin{center}
		\includegraphics[width=150pt]{app_login.png}
		\hspace{1.5 cm}
		\includegraphics[width=150pt]{app_register.png}
	\end{center}
	\caption{Login und Registrierungs Aktivität}
	\label{app_login_reg}
\end{figure}

Die Login Aktivität (Abbildung \ref{app_login_reg} links) stellt den Einstieg des Benutzers in die App dar. Grundlegend werden drei Login Varianten angeboten, wobei der Benutzer nur bei der Methode über E-Mail einen neuen Account anlegen muss, bevor er die App verwenden kann. Die Implementierung der Services von Facebook und Google erlauben es dem Nutzer sich über bereits bestehende Accounts anzumelden, ohne eine erneute Registrierung vornehmen zu müssen. \\
Außerdem besteht die Möglichkeit durch setzten eines Hakens im Feld Remember me die Daten für zukünftige Anmeldeversuche lokal in einem privaten SharedPreferences Objekt zu speichern, um beim nächsten Start der App den Login Screen zu überspringen und direkt zur Main Aktivität weitergeleitet zu werden. \\
Bei Betätigung des Login Buttons wird eine Verbindung zum Server der Datenbank aufgebaut und die in den zwei TextInput-Feldern eingegebenen Zeichen an Googles Firebase geschickt. Bei erfolgreichem Abgleich werden die Benutzerdaten vom Server heruntergeladen und der User zum Main Screen weitergeleitet. \\
Durch einen Klick auf Register Now oder Plus wird eine neue Registrierungs-Aktivität gestartet (Abbildung \ref{app_login_reg} rechts). Die Felder Name, Password und E-Mail sind verpflichtend auszufüllen. Durch Versenden einer Bestätigung-Mail wird die Validität der Adresse überprüft und der Account erst nach erfolgreicher Bestätigung erstellt.

\paragraph{Home Aktivität}
\begin{figure} [!h]
	\begin{center}
		\includegraphics[width=150pt] {app_profile.png}
		\hspace{1.5 cm}
		\includegraphics[width=150pt] {app_scan.png}
	\end{center}
	\caption{Profil und Scanner Aktivität}
	\label{app_profile_scan}
\end{figure}

Die Home Aktivität (Abbildung \ref{app_profile_scan} links) stellt dem Benutzer eine Schnittstelle zum Datenbank Server bereit, um seine Daten erstmalig hochzuladen und zu einem späterem Zeitpunkt aktualisieren zu können. Wird zum Beispiel das Gewicht verändert und eine Update Anfrage an Firebase geschickt, wird auch nur dieser Child Node der Benutzer Referenz aktualisiert und nicht das ganze Profil erneut hochgeladen. \\
Durch einen implementierten onDataChange() Listener werden serverseitige Änderungen auch direkt in Echtzeit auf der Home Aktivität aktualisiert, ohne dabei eine ständige Verbindung halten zu müssen. Die folgenden Attribute sind personalisierbar: Mail, Geschlecht, Geburtsdatum, Größe, Gewicht, Wohnort und Versicherung.\\
Durch Drücken auf das Stift-Symbol unterhalb des Profilbildes, wird ein Action Intent generiert, der die auf dem Smartphone als standardmäßige eingestellte Galerie öffnet und den Benutzer ein personalisiertes Profilbild auswählen lässt. Im onActivityResult() Callback wird dem Foto eine URI zugeordnet und in den Firebase Speicher hochgeladen. Bei einem Login über ein anderes Gerät wird nun auch das neue Profilbild angezeigt und lokal im Speicher hinterlegt, um den Internetdatenverbrauch zu minimieren. \\
Über das Dropdown Menu in der oberen rechten Ecke kann der Benutzer einen Logout Dialog starten, wobei auch die Shared Preferences Variable der Auto Login Checkbox zurückgesetzt wird. Der Button „Connect to EKG“ leitet den Benutzer zur Signal Aktivität weiter.

\paragraph{Signal  und Scanner Aktivität}
\begin{figure} [!h]
	\begin{center}
		\includegraphics[width=\textwidth] {app_signal.png}
	\end{center}
	\caption{Startseite des EKG-Gerätes}
	\label{app_signal}
\end{figure}

Die Signal Aktivität (Abbildung \ref{app_signal}) stellt die Hauptfunktionalität, das Anzeigen eines Echtzeit EKG Signals, der App bereit. Durch Drücken auf den Connect Button wird der Benutzer zur Scanner Aktivität (Abbildung \ref{app_profile_scan} rechts) weitergeleitet und aufgefordert Bluetooth zu aktivieren, falls dies nicht bereits geschehen ist. \\
Um nach Bluetooth Geräten in der Umgebung scannen zu dürfen, muss der Benutzer der App erst die Berechtigung zur Standortfreigabe erteilen, da andere Geräte dadurch auf den Ort des Geräts schließen können.
Ein Klick auf Start Scanning ruft unter anderem die startDiscovery() Methode des BluetoothAdapter Objekts auf. Sobald ein Gerät gefunden wurde wird es mit Namen und zugehöriger Mac-Adresse in ein ListView Layout eingefügt und im internen Gerätespeicher ein Abgleich gestartet, ob dieses Gerät bereits gepaired ist. \\
Durch Auswahl des EKG7-Geräts versucht die Home EKG App eine Verbindung herzustellen und gibt bei Erfolg eine Toast Nachricht aus. Die App wird die Verbindung nun solange aufrecht erhalten, bis diese vom System zwecks Inaktivität geschlossen wird oder der Disconnect Button in  der Signal Aktivität gedrückt wird. In einem neuen Thread wird der InputStream der etablierten Verbindung kontinuierlich in einen Buffer geschrieben und die Daten durch einen Handler an die Signal Aktivität weitergeleitet. Die OpenSource Grafik-Bibliothek GraphView ist für das zeichnen des Signals zuständig.