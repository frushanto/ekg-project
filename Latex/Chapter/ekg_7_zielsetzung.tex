%ekg_7 Zielsetzung

\section{Zielsetzung}

Die Diagnose eines Vorhofflimmerns ist am einfachsten in den Ableitungen Einthoven I und II möglich. Benötigt wird jedoch nur eine Ableitung, deshalb soll das EKG-Gerät lediglich ein Ein-Kanal-EKG aufzeichnen. Dieses ist prinzipiell baugleich zu einem 12-Kanal-EKG, dass im klinischen Umfeld verwendet wird, nur verfügt dieses über 12 Kanäle die parallel verschaltet sind. Für die Signalaufnahme werden Klebeelektroden mit Knopfanschluss verwendet. Nachdem das EKG-Signal durch eine analoge Filterschaltung von Störsignalen befreit wurde, wird es mit einem Mikroprozessor der Produktfamilie MSP430 von Texas Instruments, analog-digital gewandelt. Die Daten werden zur späteren Auswertung auf einer SD-Karte gesichert. Zur zusätzlichen Darstellung auf dem Mobiltelefon sollen die EKG-Daten mittels Bluetooth versendet werden. Hierfür wird eine Android-App entwickelt, die das Echtzeitsignal in einem Zeitdiagramm darstellt.

Die Bedienung erfolgt über ein Touch-Display, welches zur Darstellung des Echtzeitsignals, der Herzfrequenz und zur Information des Patienten verwendet wird. Ziel ist es die Bedienung so einfach und unmissverständlich wie möglich zu gestalten, um auch Personen ohne Fachkenntnis die Anwendung zu ermöglichen. Für die Benutzung unter mehreren Patienten eines Haushaltes, kann vor der Aufzeichnung ein Benutzerprofil auf dem Display ausgewählt werden. Zur Aufzeichnung wird das Gerät zwei verschiedene Modi bieten. 

Eine Kurzzeitaufnahme, die für Situationen geplant ist, in denen der Benutzer akut Symptome verspürt. In diesem Fall wird eine EKG-Aufnahme von zwei Minuten erstellt und mit Zeitsignaturen auf der SD-Karte gespeichert. Bei der Langzeitaufnahme wird ein EKG für 24 Stunden aufgezeichnet, um die Herzaktivität über einen längeren Zeitraum analysieren zu können. 

Während der Aufnahme soll das Gerät die Herzfrequenz aus dem Signal berechnen. Bei einem langanhaltenden Puls von weniger als 60 oder mehr als 100 Schlägen pro Minute, wird der Benutzer darauf hingewiesen, dass eine Bradykardie beziehungsweise eine Tachykardie vorliegt. Eine weiterführende automatisierte Diagnostik (z.B. von Vorhofflimmern) durch ein Programm wird nicht angestrebt, da diese Diagnose ohnehin noch einmal von einem Mediziner gestellt werden muss.  

Während des Langzeitmodus wird das Bluetooth-Modul und Display deaktiviert, um die Akkulaufzeit zu verlängern. Der Energiesparmodus kann auch außerhalb der Langzeitfunktion durch einen Taster am Gehäuse aktiviert werden. Die Energieversorgung erfolgt über eine Lithium-Polymer-Akku-Zelle, deren Ladezustand vom Mikroprozessor gemessen und auf dem Display angezeigt wird. Der Akku kann aus dem Gehäuse entnommen und extern geladen werden. Bei einem Akkustand von weniger als 20\% soll das Gerät den Nutzer einmalig mit einem akustischen Signal darauf hinweisen. 

Für die Filter- und Verarbeitungselektronik wird eine Platine entworfen, deren Herstellung bei einem externen Fertiger erfolgt. Das Gehäuse soll durch 3D-Druck hergestellt werden und besteht aus einem rechteckigen Körper und einem abnehmbaren Deckel, mit einer Aussparung für das Display.

Das Gesamtbudget für die Herstellung in der Produktion soll maximal 150 Euro betragen.