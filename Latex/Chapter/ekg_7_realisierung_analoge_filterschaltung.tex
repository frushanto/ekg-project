%ekg_7_Realisierung/analoge_Filterschaltung

\subsection{Analoge Filterschaltung}

Dieses Unterkapitel befasst sich mit den einzelnen Stufen der analogen Filterschaltung und wie die Anforderungen an sie umgesetzt werden.

\subsubsection{Eingangsstufe}

Direkt an die Klebeelektroden sind die beiden symmetrischen Eingangskanäle angeschlossen. Sie sind hochohmig um die Signalquelle nicht zu belasten. Die Kanäle bestehen jeweils aus:

\begin{enumerate}

\item einem passiven Hochpass (fg = \SI{0,48}{\hertz}) zur Abtrennung des Gleichanteils und zur Kleinsignaleinkopplung auf das Gleichspannungspotential der Filterschaltung

\item einem passiven Tiefpass (fg = \SI{159}{\hertz}) zur Unterdrückung von hochfrequenten Störungen (>\SI{100}{\kilo\hertz}) bereits vor dem Differenzverstärker

\item einer bidirektionalen TVS-Dioden zum Schutz der Schaltung vor einem ESD

\end{enumerate}

\subsubsection{Differenzverstärkung}

Zur Differenzbildung der beiden Kanäle wird ein Instrumentenverstärker von Analog Devices (AD) verwendet. Das Modell AD8422ARMZ ist ein Rail-to-Rail-Verstärker der im niedrigen Frequenzbereich bis \SI{60}{\hertz} eine Gleichtaktunterdrückung von etwa \SI{120}{\decibel} erreicht. Dies unterdrückt Störsignale die durch externe elektromagnetische Felder in die Messleitungen eingekoppelt werden. Seine Verstärkung wird mittels eines \SI{33}{\ohm}-Widerstandes auf den Faktor 420 eingestellt, was etwa in \SI{52}{\decibel} entspricht. Diese Vorverstärkung sorgt dafür, dass das Signal auf dem Weg durch die Filterschaltung robuster gegen Störungen ist. Da das EKG-Gerät über einen Akku betrieben wird, muss der Instrumentenverstärker mit einer unipolaren Versorgungsspannung arbeiten. Sein Versorgungsstrom von etwa \SI{330}{\micro\ampere} ist ebenfalls gut für einen Akkubetrieb geeignet. \cite{Datenblatt_InAmp}

\subsubsection{Kerbfilter}

Da der Körper eines Menschen aus leitendem Material besteht, können elektromagnetische Störfelder eine Spannung in ihm induzieren. Diese Wechselspannung mit einer Amplitude von bis zu \SI{100}{\milli\volt} und einer Frequenz von \SI{50}{\hertz} überlagert das EKG-Signal des Herzens. Um diese Netzschwingung zu unterdrücken, wird ein Doppel-T-Filter verwendet. Bei idealen Bauteilen erreicht diese aktive Bandsperre eine Güte von annähernd 0,5 und eine Dämpfung von \SI{76}{\decibel}. Da jedoch die verwendeten SMD-Widerstände und Kondensatoren nur mit unvermeidbaren Toleranzen erhältlich sind, fällt die effektive Dämpfung auf \SI{20}{\decibel} bis \SI{30}{\decibel}. Dies wäre für die Anwendung nicht ausreichend, daher werden in der Schaltung zwei dieser Bandsperren in Reihe geschaltet. Die im Schaltplan vorgesehenen Parallelschaltungen der Widerstände dienen dazu die Widerstandswerte flexibel einzustellen, um auch noch im Nachhinein auf die Toleranzen der Kondensatoren reagieren zu können. \cite[S. 881ff.]{Tietze_Schenk}

\subsubsection{Tiefpassfilterung}

Bei den benötigten Operationsverstärkern wurde ein Vierfach-OPV von Analog Devices gewählt. Zwei der Operationsverstärker werden für die \SI{50}{\hertz}-Filter verwendet, die anderen zwei dienen der Tiefpassfilterung und Nachverstärkung. Der AD8544ARZ ist ein Rail-to-Rail-Verstärker der mit einem geringen Versorgungsstrom von \SI{45}{\micro\ampere} auch unipolar betrieben werden kann. Da das zu filternde Signal im niederfrequenten Bereich liegt, ist er mit seinem Verstärkungs-Bandbreite-Produkt von \SI{1}{\mega\hertz} mehr als ausreichend. Mit dem Analog-Digital-Umsetzer (ADC) wurde eine Abtastfrequenz von \SI{1}{\kilo\hertz} angestrebt, die im Laufe der Entwicklung auf \SI{250}{\hertz} gesenkt wurde. Das Signal wird daher durch Tiefpässe begrenzt. Die Tiefpassfilterung setzt sich aus vier passiven Tiefpässen erster Ordnung und einem aktiven Tiefpass zweiter Ordnung zusammen. Die passiven Filter sind zwischen den aktiven Stufen der Schaltung eingebettet. \cite{Datenblatt_OPV}

\begin{figure} [!h]
	%\centering
	\includegraphics[width=\textwidth] {EKG_aktiver_Tiefpassfilter.png}
	\caption{Aktiver Tiefpass eingeschlossen von zwei passiven Tiefpässen}
	\label{aktiver Tiefpass} 
\end{figure}

In Abbildung \ref{aktiver Tiefpass} ist der verwendete aktive Tiefpass, realisiert durch eine Sallen-Key-Schaltung, abgebildet. Davor und danach befinden sich einfache passive Tiefpässe. Die zwei verbleibenden passiven Tiefpassfilter-Stufen befinden sich in der Eingangsstufe und zwischen den beiden \SI{50}{\hertz}-Filtern. Insgesamt ergibt sich damit eine Tiefpassfilterung sechster Ordnung, also eine Dämpfung von \SI{120}{\decibel} pro Dekade, über die Schaltung. \cite[S. 863ff.]{Tietze_Schenk}

\subsubsection{Nachverstärkung}

Die Signalamplitude der Quelle beträgt nur etwa \SI{1}{\milli\volt}. Der ADC arbeitet in einem Bereich von \SI{0}{\volt} bis \SI{3}{\volt}. Um diesen Bereich bestmöglich zu nutzen muss das Signal auf eine Amplitude von etwa \SI{2}{\volt} verstärkt werden. \SI{1}{\volt} der ADC-Eingangspannung bleibt als Reserve ungenutzt, um bei Schwankungen des Signals nicht sofort die Begrenzung der Spannungsversorgung zu überschreiten. Außerdem kann die Amplitude des Eingangssignals je nach Mensch auch leicht variieren. Insgesamt wird eine Verstärkung von etwa 2000 benötigt, was \SI{66}{\decibel} entspricht. Wie bereits erwähnt, wird durch den Differenzverstärker am Eingang eine Verstärkung von etwa \SI{52}{\decibel} realisiert. Die Filterstufen in der Schaltung bewirken eine Dämpfung des gesamten Signals um etwa \SI{6}{\decibel}, somit muss die Nachverstärkung \SI{20}{\decibel} betragen um die geforderte Gesamtverstärkung von \SI{66}{\decibel} zu erreichen. Dies bewirkt ein nicht-invertierender Spannungsverstärker, mit einem Verstärkungsfaktor von 10, der sein Ausgangssignal direkt auf den Pin des ADCs gibt.

\subsubsection{Frequenzübertragungsverhalten der Filterschaltung}

\begin{figure} [!h]
	%\centering
	\includegraphics[width=\textwidth] {EKG_Gesamtübertragungsfunktion.png}
	\caption{Gesamtübertragungsfunktion der Filterschaltung}
	\label{Bodediagramm Filterschaltung} 
\end{figure}

In Abbildung \ref{Bodediagramm Filterschaltung} ist die simulierte Gesamtübertragungsfunktion der Filterschaltung in einer doppelt-logarithmischen Darstellung abgebildet. Für Frequenzen kleiner als \SI{0,5}{\hertz} wird das Signal mit \SI{20}{\decibel} pro Dekade gedämpft. Ab etwa \SI{160}{\hertz} wird es durch die Tiefpässe mit \SI{120}{\decibel} pro Dekade unterdrückt. Außerdem gibt es bei \SI{50}{\hertz} eine Dämpfung von \SI{-40}{\decibel} durch die Bandsperren. Hierbei ist zu beachten, dass die Simulation mit idealen Bauteilen durchgeführt wurde. In der realen Schaltung fällt die Dämpfung wesentlich geringen aus, sodass die Übertragungsfunktion in diesem Bereich bei zirka \SI{0}{\decibel} liegt. Für den übrigen Frequenzbereich wird eine Verstärkung um \SI{67}{\decibel} erreicht. Der Gesamtschaltplan der Filterung befindet sich im Anhang. (siehe Sourcen/EKG-7 Platinendesign/Schematics\_EKG\_2021-03-09)

\subsubsection{Gleichspannungspotenzial der Filterschaltung}

\begin{figure} [!h]
	%\centering
	\includegraphics[width=\textwidth] {EKG_virtueller_Ground.png}
	\caption{Generierung des 1,5 V Bezugspotentials für die Filterschaltung}
	\label{Virtueller GND} 
\end{figure}

Um die Schaltung auf ein Gleichspannungspotenzial von \SI{1,5}{\volt} anzuheben wurde ein hochohmiger Spannungsteiler mit einem Operationsverstärker als Spannungsfolger verwendet (siehe Abbildung \ref{Virtueller GND}). Bei dem Operationsverstärker handelt es sich um den AD8541, den gleichen Verstärker der auch für die Filter zum Einsatz kommt.




