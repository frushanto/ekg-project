%ekg_7_Realisierung/Diskussion der Alternativen

\subsection{Konzeptfindung und Diskussion der Alternativen}

Dieses Kapitel führt die Anforderungen und die gewählten Lösungen auf, ohne diese im Detail zu erklären. Die genaue Erläuterung der verfolgten und getesteten Lösungen sind Inhalt der folgenden Kapitel der Realisierung. Ebenso werden hier die Lösungen aufgeführt, die sich nach dem Test als nicht zielführend für das

\subsubsection{Konzeptfindung}
%TODO genaue Abgrenzung zur Zielsetzung? 

\subsection{Verwendete Software}

Für die Erstellung und Frequenzanalyse eines künstlichen EKG-Signals wurde das numerische Rechentool Matlab verwendet. Damit konnten die Grenzfrequenzen des Signals bereits ohne Labortest abgeschätzt werden. Diese Erkenntnisse wurden bei der Schaltungsentwicklung der analogen Filterschaltung mit LTSpice angewandt. Durch die Einbindung von Herstellermodellen, war die Simulation von Bauteilen möglich, ohne diese physisch zu testen. Für den Entwurf der Leiterplatine kam Altium Designer zur Anwendung. Auch hierfür bieten Hersteller Modelle für die Pinbelegung, den Footprint und 3D-Modelle an. Besonders die 3D-Modelle waren für das Gehäusedesign hilfreich um die korrekte Lage und Maße der Bauteile im Gehäuse auch optisch zu prüfen.
%TODO weitere Software wie Nextion Editor , Git, CCS und Blender

\subsubsection{Digitale Filterung}

