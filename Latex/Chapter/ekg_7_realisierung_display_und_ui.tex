%EKG-7 Realisierung/Display und UI

\subsection{Display und Benutzeroberfläche}

Dieses Unterkapitel behandelt die Auswahl des Displays und die Programmierung der Benutzeroberfläche.

\subsubsection{Auswahl des Displays}
Gesteuert wird das EKG-Gerät durch das NX4024T032. Dabei handelt es sich um ein TFT Touch-Display der Firma Nextion. Das Display ist 3,2 Zoll groß und weist eine Auflösung von 400x240 Pixel auf. Betrieben wird das Display mit 5 V und verbraucht bei maximaler Helligkeit 85 mA. Verbunden wird das Display via UART mit einer Baudrate von 115200 $s^{-1}$. \cite{Datenblatt_Nextion}

\subsubsection{Programmierung der Benutzeroberfläche}
Der erste Schritt der Displayprogrammierung ist die Initialisierung. In diesem Schritt wird die Baudrate von 115200 $s^{-1}$ und die \textit{„Tap-to-wake“} Funktion integriert. Das Display schaltet sich je nachdem, auf welcher Seite man sich befindet, nach 30 bis 180 Sekunden in den Schlafmodus. Durch einmaliges Antippen auf das Display wird es wieder aufgeweckt. \cite{Nextion_Anweisungen}

Der Nextion-Editor verfügt über ein weites Spektrum an Werkzeugen von denen folgende im Projekt verwendet werden:
\begin{itemize}
    \item \textit{Text}: Für die Beschreibung der Funktionen und Anleitung
    \item \textit{Number}: Variablen für den \textit{Timer}, Ladestand und Herzfrequenz
    \item \textit{Button}: Zum Seitenwechsel und senden der \textit{Interrupts} an die MCU
    \item \textit{Waveform}: Für den EKG-Kurvenverlauf
    \item \textit{Slider}: Einstellung der Helligkeit
    \item \textit{Timer}: Befehle, die regelmäßig ausgeführt werden
\end{itemize}

Aufgebaut ist die Benutzeroberfläche aus verschiedenen Seiten. Nach dem Einschalten des EKG-Gerätes gelangt man auf die Startseite (siehe Abbildung \ref{fig. EKG-Startseite}). Hier ist in der oberen rechten Ecke die Peripherie zu sehen. Der Akkustand wird in Prozent angegeben und die Verbindung zur SD-Karte und zum Bluetooth-Modul überprüft. Bei einem blauen Symbol besteht eine Verbindung und bei einem grauen Symbol ist das Modul getrennt. Alle Symbole die auf dem Display vorzufinden sind, wurden aus Vorlagen an das gesamte Erscheinungsbild der Oberfläche angepasst.
\begin{figure} [!h]
	\centering
	\includegraphics[width=10cm] {ECG_Homescreen.png}
	\caption{Startseite des EKG-Gerätes}
    \label{fig. EKG-Startseite}
\end{figure}

Die sechs Symbole in der Mitte des Displays sind \textit{Buttons}, mit denen zwischen den Seiten gewechselt wird. In der Anleitung werden allgemeine Einstellungen erklärt, wie die Elektroden am Patienten anzubringen sind und welche Funktionen ein Kurzzeit- und Langzeit-EKG bietet. Beim Kurzzeit-EKG erfolgt eine zweiminütige Aufzeichnung, welche in Echtzeit auf dem Display nachverfolgt werden kann. Beim Langzeit-EKG erfolgt die EKG-Aufnahme für 24 Stunden. Die Helligkeit des Displays kann mithilfe eines Schiebereglers eingestellt werden. Optional kann ein Benutzer ausgewählt werden. Dieser Benutzer erscheint dann auf der erstellten CSV-Datei der SD-Karte. Der \textit{Standby-Modus} versetzt das Display beim Drücken in den \textit{Sleep-Modus}.

Während der Nutzung des EKG-Gerätes sind diverse Sicherheitsabfragen auf dem Display vorzufinden, wenn:
\begin{enumerate}
    \item bereits eine Aufnahme läuft und eine Andere gestartet wird
    \item eine Aufnahme mit dem Stopp-Befehl abgebrochen wird
    \item die SD-Karte entfernt wird oder beim Anschaltvorgang fehlt
    \item der Ladestand des Akkus zu Beginn einer Langzeit-Aufnahme kleiner als 80 \% ist
    \item Verdacht auf Bradykardie oder Tachykardie besteht
\end{enumerate}

Die Herzfrequenz wird während einer Kurzzeit-Aufnahme kontinuierlich berechnet. Da es sich bei diesem Projekt um ein Ruhe-EKG handelt, sollte die Herzfrequenz zwischen 60 und 100 Schläge pro Minute betragen. Im Fall von Bradykardie oder Tachykardie wird der Patient auf dem Display visuell gewarnt.