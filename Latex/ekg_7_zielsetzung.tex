%ekg_7 Zielsetzung

\section{Zielsetzung}

Die Diagnose eines Vorhofflimmerns ist am einfachsten in den Ableitungen Einthoven I und II möglich. Nötig ist jedoch nur eine Ableitung, deshalb soll das EKG-Gerät lediglich ein Ein-Kanal-EKG aufzeichnen. Dieses ist prinzipiell baugleich zu einem 12-Kanal-EKG, dass im klinischen Umfeld verwendet wird, nur verfügt dieses über 12 Kanäle die parallel verschaltet sind. Die Bedienung erfolgt über ein Touch-Display, welches auch zur Darstellung des Echtzeitsignals und zur Information des Patienten verwendet wird. Ziel ist es die Bedienung so einfach und unmissverständlich wie möglich zu gestalten, um auch Personen ohne Fachkenntnis die Anwendung zu ermöglichen. Die Daten werden zur späteren Auswertung auf einer SD-Karte gesichert. Zur zusätzlichen Darstellung auf dem Mobiltelefon sollen die EKG-Daten mittels Bluetooth versendet werden. Für die Filter- und Verarbeitungselektronik wird eine Platine entworfen, deren Herstellung bei einem externen Fertiger erfolgt.....